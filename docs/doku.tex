\documentclass[a4paper]{article}
\usepackage[ngerman]{babel}
\usepackage{amsthm}
\usepackage{amsmath}
\usepackage{amsfonts}
\usepackage{parskip}
\usepackage{graphicx}
\usepackage{color}
\usepackage{hyperref}

\hypersetup{linktoc=all,colorlinks=true,linkcolor=black}

\pagestyle{headings}

\title{Speedcube-Metrics}
\author{Aaron Schweig, Michael Angermeier, Patrick Mischka, \\ Troy Kessler, Jan Grübener}
\date{{\today}}
\begin{document}
	\maketitle
	\section{Worum geht es?}
	\subsection{Die Idee}
	\section{How to Use?}
	\subsection*{Anmeldung und Registrierung}
	Beim Starten der App wird der Nutzer zuerst gebeten, sich mit einzuloggen oder zu registrieren. 
	\begin{center}
		\includegraphics[width= 6cm, keepaspectratio]{img/login}
	\end{center}
	Er besitzt dabei die Möglichkeit einen Account mittels E-Mail Adresse und eigens gewähltem Passwort zu erstellen oder auf die Autorisierungsdienste von Google oder GitHub zuzugreifen und sich mit seinen bereits vorhandenen Accounts anzumelden. Das bei der Registrierung gewählte verfahren, muss zukünftig auch für jede Anmeldung verwendet werden.
	\subsection*{Übersicht}
	Nach dem erfolgreichen einloggen kann der sieht der Nutzer eine Auswertung seiner letzten Ergebnisse. Hier wird nach Trainings und Wettkämpfen unterschieden und die Ergebnisse des vergangenen Monats graphisch aufgezeigt. So lassen sich Erfolge und Tendenzen schnellstmöglich erkennen.
	\begin{center}
		\includegraphics[width= 6cm, keepaspectratio]{img/overview}
	\end{center} 
	Um schnellstmöglich mit dem Spiel beginnen zu können, kann gibt es direkt unter der graphischen Auswertung Buttons um ein Training oder Wettkampf zu starten. Daten interessierte können die Möglichkeit nutzen, sich mit der Hilfe des Übersicht Buttons genauer in ihre erreichten Resultate einzulesen und so sich weiter zu verbessern. 
	\begin{center}
		\includegraphics[width= 6cm, keepaspectratio]{img/competitionoverview}
	\end{center}
	Dafür wird ihnen extra ein eine Tabelle mit den vergangenen Resultaten angezeigt. Für abgeschlossene Trainings wird die jeweilige Lösungszeit des Scrumbels angezeigt. Für absolvierte Wettkämpfe wird hier der Average 3 angezeigt. Über die Spalte Analyse können die Ergebnisse der einzelnen Runden eines Wettkampfes sowieso weitere Daten wie Average 5, Best, Worst, etc. angezeigt werden.
	\begin{center}
		\includegraphics[width= 5cm, keepaspectratio]{img/competitiondetails}
	\end{center}  
	\subsection*{Navigation}
	Die Navigation innerhalb der App kann entweder durch die in den einzelnen Abschnitten eingepflegten Buttons erfolgen, die mit einer einfachen Beschreibung zu den logischen nächsten Schritten führen oder über die im Menü eingebundene Navigationsleiste.
	\begin{center}
		\includegraphics[width= 5cm, keepaspectratio]{img/nav}
	\end{center}
	Die Navigationsleiste ermöglicht jederzeit das schnelle wechseln zu den Übersichten der letzten Trainings und Wettkämpfe sowie das Starten dieser. Hier findet der Nutzer auch die Möglichkeit vor sich auszuloggen, bevor er die App beendet.  
	\subsection*{Training}
	Ein Training beginnt immer damit, dass die App einen automatisch generierten Scrumpel vorschlägt, welcher im Anschluss der Nutzer auf seinem Rubik´s Cube übernimmt. Dies kann er entweder über den als Text angezeigten Scrumbel oder mithilfe des 3 Dimensional dargestellten Cubes. Zur einfacheren Handhabung und Kontrolle ist dieser durch halten und Bewegungen im Raum dreh und bewegbar. Unpassende Scrumbels können mit dem "Neuer Scrumble" Button einfach übersprungen werden.
	\begin{center}
		\includegraphics[width= 5cm, keepaspectratio]{img/training}
	\end{center} 
	Der Start Button startet automatisch den eingeblendeten Timer. Wird man bei der Lösung des Scrumbels gestört oder möchte aus anderen Gründen einen neuen Versuch starten, so kann der aktuelle Durchgang durch den Abbrechen Button unterbrochen und der Scrumbel erneut gelöst werden. Stop speichert das Training und bietet sofort einen neuen Scrumbel für einen neuen Versuch. 
	\begin{center}
		\includegraphics[width= 5cm, keepaspectratio]{img/trainingdone}
	\end{center}
	\subsection*{Wettkampf}
	Ein Wettkampf besteht immer aus 5 unterschiedlichen Scrumbels, die schnellstmöglich hintereinander gelöst werden sollen. Das drücken des Stopp Buttons zeigt immer automatisch den nächsten Scrumbel an, der mit dem eigenen Rubiks Cube gelöst werden soll. Hierfür muss der Timer wieder seperat gestartet und gestoppt werden. 
	\begin{center}
		\includegraphics[width= 5cm, keepaspectratio]{img/competition}
	\end{center} 
	Nach dem alle 5 Scrumbels gelöst wurden, bekommt der Nutzer die Möglichkeit den Wettkampf zu verwerfen oder ihn zu speichern. Das Verwerfen führt automatisch zu einem neuen Wettkampf mit anderen Scrumbels, das speichern zeigt die Ergebnisse des Wettkampfes an und berechnet wichtige Daten wie Average 4 und Avergage 5 um sich mit alten Wettkämpfen oder Freunden vergleichen zu können.  
	\begin{center}
		\includegraphics[width= 5cm, keepaspectratio]{img/competitiondone}
	\end{center} 
	\section{UI Design}
	\section{Technology \& Architecture}
	\section{Testing}
	\section{Backlog}
\end{document}