% !TeX root = ./doku.tex

Im folgenden Abschnitt wird kurz die Wahl eines \textbf{BaaS} (Backend as a Service) Anbieters erläuter. Dabei liegt der Fokus vor allem darauf, wieso ein BaaS unter dem gegebenen Anwendungsfall sinnvoll ist.

\subsection{Was ist ein BaaS}
\begin{definition}[Backend as a Service]
    Backend-as-a-Service (BaaS) ist ein Cloud-Service-Modell, bei dem Entwickler alle Hintergrundaspekte einer Web- oder Mobilanwendung auslagern, so dass sie nur das Frontend schreiben und warten müssen. BaaS-Anbieter bieten vorab geschriebene Software für Aktivitäten auf Servern an, z. B. Benutzerauthentifizierung, Datenbankverwaltung, Remote Updates und Push-Benachrichtigungen (für mobile Apps) sowie Cloud-Storage und -Hosting
    \cite{cloudfare}.

    Anhand dieser Definition lässt sich erkennen, dass ein BaaS darauf abzielt, Entwicklungszeiten zu verkürzen und klassische Aufgaben bei der Erstellung einer Webapp zu standardisieren. Zu dem typischen Elementen, die durch BaaS-Anbieter übernommen werden gehören:
    \begin{itemize}
        \item Datenbankverwaltung
        \item Cloud-Speicher
        \item Authentifizierung
        \item Push-Notifications
        \item Hosting
    \end{itemize}

    Es ist zu erkennen, dass so die Entwicklung und die initiale Hürde verloren geht, da elementare Bestandteile einer Webapplikation bereites vorhanden sind und nicht für jeden Anwendungsfall erneut entwickelt werden müssen. Mithilfe einer vorhanden Datenbank und einer sicheren, verwalteten Nutzerauthentifizierung sind bereits Grundsteine gelegt, damit eine Webapp erfolgreich aufgebaut werden kann. Das Modell eines BaaS ist noch nicht sehr alt und wurde innerhalb der letzten Jahre immer populäre. Ereignisse, wie die strikte Regulierung der DSGVO stellt die Nutzung dieser Modelle in einem Enterpreiseumfeld vor Herausforderungen. Diese müssen vor einer Entscheidung für einen solchen Dienst abgewogen werden.

    \subsection{Wieso Firebase?}
    Innerhalb dieses Anwendungsfalles lassen sich verschiedene Aspekte finden, welche für die Nutzung eines BaaS und auch Firebase als populären Anbieter eines solchen Service sprechen.
    \begin{enumerate}
        \item \textbf{Speicherung weniger bis garkeiner nutzerspezifischer Daten:}\\
              In diesem Anwendungsfall werden Nutzerdaten nur gespeichert, damit der eigene Fortschritt visualisiert werden kann. Das bedeutet im Kern, dass außer einer eindeutigen Mailadresse keine zusätzlichen Informationen eines Nutzers benötigt werden. Mithilfe von verschiedenen Social-Auth Providern kann auch eine sichere Authentifizierung gewährleistet werden, da diese immer einem aktuellen Web-Standard - OAuth 2.0 - folgen.
        \item \textbf{kaum speziell im Backend auszuführende Logik:}\\
              Dadurch, dass diese Applikation vorerst rein zum visualisieren des persönliches Fortschritts dient, gibt es sehr wenig Logik, die auf einem Server ausgeführt werden müssten. Durch die direkte Verbindung zwischen Frontend und Datenbank mittels Firebase können Lese- und Schreibanforderungen direkt im Frontend gesteuert werden. Die Logik zum generieren eines zufälligen Scrumbles und zum erstellen des 3D-Cubes wird mithilfe einer serverless Function \footnote{Der Begriff „Serverless" (serverlos) bezieht sich auf ein cloudnatives Entwicklungsmodell, bei dem Entwickler Anwendungen erstellen und ausführen können, ohne Server verwalten zu müssen. \cite{redhat}} ausgeführt.
        \item \textbf{einfache Kommunikation zwischen Frontend und Datenpersistierung ohne Schnittstellenprobleme:}\\
              Mithilfe von Firebase ist es möglich, wie bereits oben beschrieben, direkt aus dem Frontend mit der Datenbank zu interagieren. Dabei werden natürlich alle Sicherheitsstandards seitens Firebase eingehalten, damit es nicht zu einem Missbrauch der Daten kommen kann. Während der Entwicklung einer Applikation steht ein Team oft vor dem Problem der Schnittstellendefinition. Dabei gilt es vor allem folgende Fragen zu beantworten:
              \begin{enumerate}
                  \item Welches Format wird verwendet (REST, GraphQL, SOAP, etc.)?
                  \item Welche Struktur haben die Daten?
              \end{enumerate}
              Sind diese Fragen beantwortet, so muss zuerst ein gesamtes Backend entwickelt werden, in welchem die Funktionalität unter Berücksichtigung vieler Sicherheitsaspekte oder Best-Practices umgesetzt werden müssen. Das kostet Zeit und bietet ein hohes Potenzial für menschenbedingte Fehler. Zum Beispiel können Endpunkte nicht funktionieren, es bestanden unterschiedliche Ansichten bzgl. der Nutzung und des Funktionsumfangs des gewählten Schnittstellenformats oder gewünschte Operationen konnten nicht rechtzeitig implementiert werden.

              All diese Probleme sind durch die direkte Kommunikation zwischen den UI-Elementen und der Datenpersistierung gelöst. Es kann sich voll und ganz auf das sinnvolle Erstellen einer Datenstruktur konzentiert werden, sodass am Ende auch hier keine Missverständnisse zwischen verscheidenen Teams entstehen. Im Falle der hier vorgestellten Applikation erleichtert uns das die Entwicklung enorm, weswegen auch unter Berücksichtigung eines Zeitfaktors viele Vorteile entstehen.
    \end{enumerate}
\end{definition}
